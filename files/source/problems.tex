\documentclass[12pt]{article}

\usepackage{amsmath}
\usepackage[dvips]{graphicx}
\usepackage{psfrag}
\usepackage{fancyhdr}
\usepackage{lastpage}
\usepackage[
  colorlinks=true,
  urlcolor=blue
  ]{hyperref}

\topmargin       -0.5truein
\headsep          0.5truein
\headheight       0.0truein
\oddsidemargin   -0.4truein
\evensidemargin  -0.4truein
\textwidth        6.5truein
\textheight       9.0truein
\addtolength{\hoffset}{1.0cm}

\pagestyle{fancyplain}
\chead{\small EPFL MSE-468-Quantum Simulations of Materials-2018}
\cfoot{Page~\thepage~of~\pageref{LastPage}}

\def\QE{\textsc{Quantum ESPRESSO}}

\begin{document}


  $ $
  \vspace{5mm}

  \begin{center}

  { \bf \Large Lab 2: Assignment\\ First-principles calculations for bulk NaCl} \\

  \end{center}
  \vspace{5mm}

  Use the \QE\, package to solve these problems.\\ 

  IMPORTANT: Report in the solutions not only the final plots, but also all the relevant input file parameters and the formulas that you have derived in order to solve the exercises.\\

  IMPORTANT: For problems from 1 to 7 use the primitive unit cell with 2 atoms as shown in the example. For problem 8 use a conventional unit cell with 8 atoms.  
  \vskip 0.4 cm

  \noindent
  {\bf Problem 1} (10 points):
  Convergence of {\em total (absolute) energies} with respect to cutoff energies.

  \begin{itemize}

    \item[A]
    Calculate the total energy of bulk NaCl as a function of the kinetic-energy cutoff.
    A good range to try is 10-150 Ry, doing calculations at increments of 10 Ry.
    When changing the cutoff, make sure to keep the other variables (lattice constant, $\mathbf{k}$ points mesh, etc.) fixed and to record them.
    Record and plot your final results.
    Specify when you reach the level of convergence of around 5 meV/atom (convert
    this to Ry/atom). Note that PWscf calculates energy per unit cell.

    \item[B]
    Do you see a trend in your calculated energies with respect to cutoff?
    If you see a trend, is this what you expect and why?
    If not, why?

    \item[C]
    In Problem Set 1 during the lab, you used a cubic cell. Here, we use the primitive cell.
    What are the advantages and disadvantages of both methods?

  \end{itemize}
  \vspace{6mm}

  %%%%%%%%%%%%%%%%%%%%%%%%%%%%%%%%%%%%%%%%%%%%%%%%%%%%%%%%%%%%%%%%%%%%%%%%%%%%%%%%

  \noindent
  {\bf Problem 2} (10 points):
  Convergence of {\em forces} with respect to cutoff energies. \\

  \noindent
  In some cases, we are interested in quantities other than energies.
  In this problem, we will be calculating forces acting on atoms.
  Displace a Na (or Cl) atom by $+0.05$ in the $z$ direction (fractional coordinates).
  Keeping other parameters fixed, calculate the forces on a Na (or Cl) atom as a
  function of cutoff.
  Reach the convergence on forces to within around 10 meV/A
  (convert this to Ry/Bohr, since PWscf gives forces in Ry/Bohr).
  Record relevant parameters. Use the $\mathbf{k}$ points mesh fixed to $4 \times 4 \times 4$ centered at the Brillouin zone.
  Plot and record your results.
  \vspace{10mm}


  %%%%%%%%%%%%%%%%%%%%%%%%%%%%%%%%%%%%%%%%%%%%%%%%%%%%%%%%%%%%%%%%%%%%%%%%%%%%%%%%

  \newpage

  \noindent
  {\bf Problem 3} (10 points):
  Convergence of the {\em total (absolute) energies} with respect to the size of the $\mathbf{k}$ points mesh.

  \begin{itemize}

    \item[A]
    Calculate the total energy as a function of $\mathbf{k}$ points mesh size.
    For each mesh, record the number of the $\mathbf{k}$ points in the irreducible wedge of the first Brillouin zone.
    This gives a measure of how long your calculation will take --- calculations
    scale linearly with the number of $\mathbf{k}$ points.
    When changing the size of the $\mathbf{k}$ points mesh, make sure to keep all other
    input parameters fixed (lattice constant, energy cutoff, etc.).

    \item[B]
    Do you see a trend in your calculated energies with respect to size of the $\mathbf{k}$ points mesh? If you see a trend, is this what you expect and why? If not, why?

  \end{itemize}
  \vspace{6mm}



  %%%%%%%%%%%%%%%%%%%%%%%%%%%%%%%%%%%%%%%%%%%%%%%%%%%%%%%%%%%%%%%%%%%%%%%%%%%%%%%%

  \noindent
  {\bf Problem 4} (10 points):
  Convergence of {\em forces} with respect to the size of the $\mathbf{k}$ points mesh. \\

  \noindent
  Calculate the force acting on a Na (or Cl) atom displaced by $+0.05$ in the $z$
  direction (in fractional coordinates) as a function of size the $\mathbf{k}$ points mesh.
  Keep all other parameters fixed. Record all relevant input parameters (lattice parameter, energy cutoff, etc.).
  Reach the convergence on forces to within around 10 meV/A
  (convert this to Ry/Bohr, since PWscf gives forces in Ry/Bohr).
  \vspace{10mm}

  %%%%%%%%%%%%%%%%%%%%%%%%%%%%%%%%%%%%%%%%%%%%%%%%%%%%%%%%%%%%%%%%%%%%%%%%%%%%%%%%

  \noindent
  {\bf Problem 5} (5 points):
  Convergence of the {\em total energy differences} with respect to energy cutoff. \\

  \noindent
  In practice only energy differences have physical meaning, as opposed to
  absolute energy scales, which can be arbitrarily shifted.
  Therefore, it is important to understand the convergence properties.
  Calculate the total energy difference between two crystals
  at different lattice parameters, as a function of cutoff.
  For example, you could calculate the energy of NaCl at the experimental
  lattice parameter (5.640 Angstrom), and then calculate the energy using another
  value close to it (5.644 Angstrom, for example), take the difference between the
  two energies, and repeat for many energy cutoffs.
  Make sure to keep all other input parameters (lattice constant, $\mathbf{k}$ points mesh, etc.)
  fixed while changing the energy cutoff.
  Record all relevant parameters such as the lattice constant, $\mathbf{k}$ points mesh, etc.
  Keep increasing the energy cutoff until you reach the convergence to around 5 meV/atom (convert this to
  Ry/atom).
  \vspace{10mm}

  %%%%%%%%%%%%%%%%%%%%%%%%%%%%%%%%%%%%%%%%%%%%%%%%%%%%%%%%%%%%%%%%%%%%%%%%%%%%%%%%

  \noindent
  {\bf Problem 6} (5 points): Comparing Problems 1, 2, 3, 4, and 5. \\

  \noindent
  How do the energy cutoff requirements change when looking at total energies,
  looking at forces, and looking at total energy differences?
  How do the $\mathbf{k}$ points mesh requirements change?
  Can you explain this?
  \vspace{10mm}

  %%%%%%%%%%%%%%%%%%%%%%%%%%%%%%%%%%%%%%%%%%%%%%%%%%%%%%%%%%%%%%%%%%%%%%%%%%%%%%%%

  \newpage

  \noindent
  {\bf Problem 7} (25 points): Determination of the {\it equilibrium lattice parameter} and {\it bulk modulus}. \\

  \noindent
  Usually, one is interested in quantities such as forces and energy differences.
  For this reason, use the energy cutoff and $\mathbf{k}$ points mesh which you determined for
  the force and energy difference calculations.\footnote{In general, to be absolutely safe you should test the convergence of the quantity you are interested in (lattice parameter and bulk modulus in this case) with respect to the energy cutoff and $\mathbf{k}$ points mesh (and other parameters, which we don't have for our system (e.g. smearing)). But we are not going to do this.}

  \begin{itemize}

    \item[A]
    Calculate the equilibrium lattice parameter of NaCl. How does the theoretical (computed) equilibrium lattice parameter $a_0^{theor}$ compares with the experimental equilibrium lattice parameter $a_0^{exp} = 5.640$ (A)? Is this expected? Make sure to record all the relevant input parameters of the calculations (energy cutoff, $\mathbf{k}$ points mesh, etc.).

    \item[B]
    Calculate the bulk modulus $B$ of NaCl. The bulk modulus is a measure of the stiffness of a material. It is defined as
    $$ B = - V_0 \frac{\partial P}{\partial V}, $$
    where $P$ is the pressure on the material, $V$ is its volume, and $V_0$ is its
    equilibrium volume. You have to derive some (simple) equations and then apply them to solve a problem. Remember that $P=-\partial{E}/\partial{V}$ (the PWscf program calculates energies per unit cell). 
    Write down your derivation in your report.

    \item[C]
    Calculate the bulk modulus $B$ of NaCl using the third-order Birch-Murnaghan isothermal equation of state. We suggest to use the provided interactive {\bf ev.x} program provided with the \QE, 
    which contains the implementation of this equation. In the input file for ev.x you have to provide two columns for the case of an ffc lattice: 
    the first one contains the lattice parameter and the second one the total energy obtained. The 
    fitting function is labeled as "birch2". If you use
    other strategies for fitting the Birch-Murnaghan isothermal equation of state specify in the solution.

    \item[D]
    Compare your computed bulk modulus, obtained with two methods described above, between themselves and with the experimental value $B_{exp}=24.42$ GPa. Which conclusions can you make? Explain your conclusions.

  \end{itemize}

  %%%%%%%%%%%%%%%%%%%%%%%%%%%%%%%%%%%%%%%%%%%%%%%%%%%%%%%%%%%%%%%%%%%%%%%%%%%%%%%%

  \noindent
  {\bf Problem 8} (25 points):\\

  \noindent
  Calculate the elastic constants $C_{11}$, $C_{12}$, and $C_{44}$ of NaCl. Use the conventional unit cell containing 8 atoms. 
  You will need to compute the energetics of deformation, and fit the resulting energy curves. Make the calculations in two ways [A] and [B].

  \begin{itemize}

    \item[A]
    Compute $C_{11}-C_{12}$ using the orthorombic ({\tt ibrav=8}) Bravais lattice, and using the relation $B = \frac{1}{3} (C_{11} + 2 C_{12})$ determine the elastic constants $C_{11}$ and $C_{12}$. Then compute $C_{44}$ using the monoclinic ({\tt ibrav=12}) Bravais lattice.

    \item[B]
    Compute $C_{11}-C_{12}$ using {\tt ibrav=0} and specifying the card {\tt CELL\_PARAMETERS}, and then use the relation for $B$ (as above) to determine the elastic constants $C_{11}$ and $C_{12}$. Compute $C_{44}$ using {\tt ibrav=0} too.

    \item[C]
    Do you obtain the same results in [A] and [B]? Compare your results with the \href{https://doi.org/10.1016/S0022-3697(72)80468-2}{experimental data} 

    %H.~J.~McSkimin, Jr.~P.~Andreatch, and P.~Glynn, ``The Elastic Stiffness of Diamond'', J. Appl. Phys., {\bf 43}, 985 (1972). Which conclusions can you make? Explain your conclusions.

  \end{itemize}
  IMPORTANT: for this exercise you can use the converged parameters that you estimated in problems 1-7. In principle, you should need less $k$ points with the conventional supercell to get to convergence, but 
  you are not required to test again.

\end{document}
